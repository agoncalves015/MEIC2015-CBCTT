\chapter{State of the Art}
\label{overview}
\thispagestyle{plain}
As mentioned previously, the automated construction of timetables is a research field for which extensive literature exists. Carter \cite{Carter1986} proposed a categorization of four types of methods that can be used to solve these type of problems: sequential methods, cluster methods, constraint-based methods and generalized search. Petrovic et al. ~\cite{Petrovic2004} proposed additional methods, such as hybrid evolutionary algorithms, meta-heuristics, multi-criteria approaches, hyper-heuristics and adaptive approaches.\\
\\
Socha et al. ~\cite{Socha2002} proposed an approach to tackle the enrollment-based course timetabling problem that applied local search and ant colony optimization strategies. Rossi-Doria et al. ~\cite{Rossi-Doria2003} proposed the usage of evolutionary algorithms to solve the timetabling problem and presented a comparison between this method and several meta-heuristic methods.
Burke et al. ~\cite{Burke2003} applied a tabu-based hyper-heuristic to the university course timetabling problem. Burke et al. ~\cite{Burke2007} proposed the use of tabu search along with a graph based hyper-heuristic. Abdullah et al. ~\cite{Abdullah2005} developed a variable neighborhood search approach in conjunction with a tabu list.\\
McMullan ~\cite{McMullan2007} proposed the application of an extended great deluge algorithm and Landa-Silva et al. ~\cite{Landa-Silva2008} presented another variation of the great deluge algorithm, the non-linear great deluge.\\
The combination of genetic algorithm and local search has been previously employed by Abdullah et al. ~\cite{Abdullah2008}.\\
\\
Müller ~\cite{Mueller2007} presented a constraint-based solver to the curriculum-based course timetabling problems in the 2nd International Timetabling Competition, ITC2007 (Track 1 and Track 3) and achieved the first place in this competition. Lü et al. ~\cite{Lue2010} applied a hybrid heuristic algorithm called Adaptive Tabu Search (ATS) to the same instances.\\
Clark et al. ~\cite{Clark2008} applied repair-based heuristic search on Track 3 datasets in the ITC2007 timetabling competition. Geiger ~\cite{Geiger2008} applied a stochastic neighborhood method based on threshold acceptance criteria to overcome the local optima to the same instances. Atsuta et al. ~\cite{Atsuta2007} applied the constraint satisfaction problem (CSP) which implemented a hybridization of tabu search and iterated local search algorithms to handle weighted constraints.\\
De Cesco et al. ~\cite{DeCesco2008} applied a dynamic tabu search to curriculum-based course timetabling, a short term tabu exclusion with variable size tabu length, with dynamic weight adjustment for hard and soft constraints.\\
Lach et al. ~\cite{Lach2008} applied an integer programming method in order to create a solver to handle timetabling problem instances.\\
Burke et al. ~\cite{Burke2009} proposed a solver based on a hybrid meta-heuristic to tackle scheduling problems.\\
\\
Surveys that document these and additional approaches that handle these type of problems, can be consulted in works such as Carter et al. ~\cite{Carter1996}, Burke et al. ~\cite{Burke2002}, Carter ~\cite{Carter1986}, and Schaerf ~\cite{Schaerf1999}.
