\chapter{Curriculum-Based Course Timetabling problem}
\label{cbcttproblem}
\thispagestyle{plain}

In this section we present the formulation of the CB-CTT problem.
The specific problem of Curriculum-Based Course Timetabling, as proposed by Di Gaspero et al. ~\cite{DiGasperoL.2007}, is an optimization problem, which consists of the weekly scheduling of the lectures for several university courses within a given number of rooms and time periods, where conflicts between courses are set according to the curricula published by the University and not on the basis of enrolment data.\\
\\
The problem consists of the following entities:\\
\\
\begin{itemize}
\item Days, Timeslots, and Periods: Given a number of teaching days in the week (typically 5 or 6) and considering that each day is split in a fixed number of timeslots, which are equal for all days. A period is considered to be a pair composed by a day and a timeslot. The total number of scheduling periods is the product of the days times the day timeslots. 
\item Courses and Teachers: Each course consists of a fixed number of lectures to be scheduled in distinct periods, is attended by given number of students, and is taught by a teacher. For each course there is a minimum number of days that the lectures of the course should be spread in, moreover there are some periods in which the course cannot be scheduled.
\item Rooms: Each room has a capacity, which corresponds to the number of available seats. All rooms are equally suitable for all courses.
\item Curricula: A curriculum is a group of courses such that any pair of courses in the group has students in common. Based on curricula, we have the conflicts between courses and other soft constraints. 
\end{itemize}
A feasible solution of the problem will correspond to an assignment of a period (day and timeslot) and a room to all lectures of each course and must conform to the following Hard constraints:\\
\\
\begin{itemize}
\item Lectures: All lectures of a course must be scheduled and they must be assigned to distinct periods. 
\item Room Occupancy: Two lectures cannot take place in the same room and in the same period. 
\item Conflicts: Lectures of courses in the same curriculum or taught by the same teacher must be all scheduled in different periods.
\item Availabilities: If the teacher of the course is not available to teach that course at a given period, then no lectures of the course can be scheduled at that period.
\end{itemize}
Additionally, the following Soft constraints should be met as well as possible:\\
\\
\begin{itemize}
\item Room Capacity: For each lecture, the number of students that attend the course must be less or equal than the number of seats of all the rooms that host its lectures. 
\item Minimum Working Days: The lectures of each course must be spread into a minimum number of days. 
\item Curriculum Compactness: Lectures belonging to a curriculum should be adjacent to each other (i.e., in consecutive periods). For a given curriculum we account for a violation every time there is one lecture not adjacent to any other lecture within the same day. 
\item Room Stability: All lectures of a course should be given in the same room. 
\end{itemize}
A solution that conforms to these Soft Constraints is one that exhibits desired structural properties like coherent daily time slots for lectures of the same curriculum, etc.