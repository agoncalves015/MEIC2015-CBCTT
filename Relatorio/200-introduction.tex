\chapter{Introduction}
\label{introduction}
\thispagestyle{plain}

Every year, universities around the world, at the beginning of each term face the problem of creating a schedule or timetable for the subjects lectured. These timetables must satisfy the needs of a university and at the same time, attempt to satisfy as many of the wishes and requirements of both the university staff and students. \\
\\
Elaboration of these timetables done manually by administrative staff of the university is a time-consuming task which is both difficult and prone to error. \\
\\
Historically the timetabling problem was considered a subject of the field of operational research, but in the last decades, it has also become a subject of research in the field of Artificial Intelligence, more specifically the application of \emph meta-heuristic algorithms to solve these type of NP-hard problems ~\cite{Cooper1996}, ~\cite{Even1976} (e.g. \emph{constraint satisfaction}, \emph{tabu search} and \emph{genetic algorithms} are a few of the techniques employed). \\
\\
Although there is a lot of literature published regarding proposed solutions that are effective at solving these problems, after reading through some of these proposals, we infer that the formulation of the problem to solve is different in each proposal and usually adjusted to the rules and specific requirements of a given institution and the results obtained in the implementations of these solutions reflect this tight coupling. \\
\\
The International Timetabling Competition (ITC)~\cite{McCollum} was established to stimulate interest in the general area of educational timetabling while providing researchers with models of the problems faced which incorporate an increased number of real world constraints.\\
\\
This competition was divided in three tracks:\\
\begin{itemize}
\item Track 1 - Examination Timetabling
\item Track 2 - Post Enrollment Based Course Timetabling (PE-CTT)
\item Track 3 - Curriculum Based Course Timetabling (CB-CTT)
\end{itemize}
For each track, it was provided the common generic formulation of the problem to solve, which in this case is the creation of feasible timetables and the hard and soft constraints that a proposed solution must observe. In order to compare the results obtained with the proposed solution solvers, common datasets, representing real-word instances that represent the problem to solve, were provided, in order to give researchers a common benchmark.\\
\\
%%%%%%%%%%%%%%%
\\
The goal of this project is to implement a solver program for the problem of Curriculum Based Course Timetabling as it was formulated in the Track 3 ~\cite{McCollum2010} of the second ITC competition that took place in 2007.\\
\\
For this solver program, we propose the implementation of a Hybrid system, in which in the first stage, a genetic algorithm is used to obtain a set of feasible solutions. In a second stage, the results obtained are processed by a local search algorithm – i.e. Simulated Annealing – in an attempt to further improve the solutions, in terms of Soft Constraints.\\
\\
%%%%%%%%%%%%%%%
The resulting implementation will be tested with the ITC 2007 datasets, which in the case of track 3 corresponds to real datasets that were obtained from university of Udine in Italy, and under some of the rules proposed in this competition. The results obtained will be compared both between the implemented s and the results of the solutions proposed by the winner of the ITC 2007 Track 3 ~\cite{Muller},~\cite{Muller2009}, in order to evaluate if the implemented solver can provide both better quality solutions and/or can be faster at obtaining these solutions.\\
\\
%%%%%%%%%%%%%%%
This document is divided in the following Subsections:
\begin{enumerate}
\item Curriculum-Based Course Timetabling Problem: In this section, we present the problem of the Curriculum-Based Course Timetabling.
\item State of the Art: In this section we present an overview of several proposed approaches that deal with the timetabling problem.
\item Proposed solution: In this section we present the selected algorithms that will later be implemented. We propose a hybrid solution that employs a Genetic Algorithm to obtain a set of feasible solutions that will be further improved through the usage of local search local search algorithms, more specifically the Simulated Annealing algorithm.
\item Project Development Planning: In this section, we present the overall project plan.
\item Future Work: In this section, we present some project topics that should be considered in the future.
\end{enumerate}
\let\cleardoublepage\clearpage