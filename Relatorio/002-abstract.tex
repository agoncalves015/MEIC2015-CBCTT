\chapter*{Acknowledgments}
\addcontentsline{toc}{chapter}{Acknowledgments} 
I wish to express my apreciation to Artur Ferreira and Nuno Leite, for their continous support and orientiation throughout this project. Whithout  their input and valuable advice, I would not be able to finish this project.\\
\\
I would also like to thank my family, for all the support and motivation that they provided throughout my life.\\
\\
I would also like to thank the Instituto Superior de Engenharia de Lisboa, for allowing me to complete this academic goal of mine.


\chapter*{Abstract}
\addcontentsline{toc}{chapter}{Abstract} 
Every year, universities around the world, at the beginning of each term face the problem of creating a schedule or timetable for the subjects lectured. These timetables must satisfy the needs of a university and, at the same time, attempt to satisfy as many of the wishes and requirements of both the university staff and students. \\
\\
Historically the timetabling problem was considered a subject of the field of operational research, but in the last decades, it has also become a subject of research in the field of Artificial Intelligence, more specifically the application of metaheuristic-algorithms to solve these type of NP-complete problems (e.g. constraint satisfaction, tabu search and genetic algorithms are a few of the techniques employed). \\
\\
The goal of this project is to implement a solver program for the problem of Curriculum Based Course Timetabling as it was formulated in the Track 3 of the second ITC competition that took place in 2007.\\
\\
This solver program will be a hybrid system consisting in the application of a genetic algorithm to obtain feasible solutions in an early stage. The quality of these solutions further improved by the usage of a local strategy, namely a Simulated Annealing algorithm. 

\section*{Keywords}
Timetabling; Curriculum-Based Course Timetabling; Metaheuristics; Evolutionary Algorithms; Simulated Annealing; Local Search; Java 
