\chapter*{Resumo}
\addcontentsline{toc}{chapter}{Resumo} 
Todos os anos, universidades de todo o Mundo, no início de cada período letivo deparam-se com o problema da criação de horários para as diversas aulas,
Estes horários têm de satisfazer as necessidades da instituição e, ao mesmo tempo, satisfazer as necessidades quer dos docentes, quer dos alunos.\\
\\
Históricamente o problema da criação automática de horários é uma área de pesquisa do domínio da investigação operacional, contudo, nas últimas décadas, tornou-se uma área de investigação no campo de Inteligência Artificial, mais concretamente, a aplicação de meta-heuristícas na resolução deste tipo de problemas NP-hard (e.g. como é o caso de algoritmos genéticos, pesquisa tabu etc.)\\
\\
O objetivo deste projeto consiste na implementação de um programa "solucionador" para o problema do Curriculum Based Course Timetabling, tal como foi formulado na Track 3 da segunda competição do International Timetabling Competition, que ocorreu em 2007.\\
\\
Este programa "solucionador" será um sistema hibrído, que consiste, aplicação de um algoritmo genético, que permita obter soluções "fazíveis", numa fase inicial. A qualidade das soluções obtidas nesta fase, será posteriormente melhorada, através da aplicação de um algoritmo que implementa uma estratégia de pesquisa local, mais concretamente, um algoritmo do tipo Simulated Annealing.\\
\\
Após a implementação do programa, iremos efectuar uma análise comparativa entre esta solução e soluções implementadas por outros autores.\\
\chapter*{Abstract}
\addcontentsline{toc}{chapter}{Abstract} 
Every year, universities around the world, at the beginning of each term face the problem of creating a schedule or timetable for the subjects lectured. These timetables must satisfy the needs of a university and, at the same time, attempt to satisfy as many of the wishes and requirements of both the university staff and students. \\
\\
Historically the timetabling problem has been considered a subject of the field of operational research, but in the last decades, it has also become a subject of research in the field of Artificial Intelligence, more specifically the application of metaheuristic-algorithms to solve these type of NP-hard problems (e.g. constraint satisfaction, tabu search and genetic algorithms are a few of the techniques employed). \\
\\
The goal of this project is to implement a solver program for the problem of Curriculum Based Course Timetabling as it was formulated in the Track 3 of the second International Timetabling Competition competition that took place in 2007.\\
\\
This solver program will be a hybrid system consisting in the application of a genetic algorithm to obtain feasible solutions in an early stage. The quality of these solutions are further improved by the usage of a local search strategy algorithm, namely a Simulated Annealing algorithm.\\
\\
After the implementation of the solver, we will conduct a comparative analysis between this solution and solutions implemented by other authors.\\

\section*{Keywords}
Timetabling; Curriculum-Based Course Timetabling; Metaheuristics; Evolutionary Algorithms; Simulated Annealing; Local Search; Java 
